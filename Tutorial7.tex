% Options for packages loaded elsewhere
\PassOptionsToPackage{unicode}{hyperref}
\PassOptionsToPackage{hyphens}{url}
%
\documentclass[
]{article}
\usepackage{amsmath,amssymb}
\usepackage{lmodern}
\usepackage{iftex}
\ifPDFTeX
  \usepackage[T1]{fontenc}
  \usepackage[utf8]{inputenc}
  \usepackage{textcomp} % provide euro and other symbols
\else % if luatex or xetex
  \usepackage{unicode-math}
  \defaultfontfeatures{Scale=MatchLowercase}
  \defaultfontfeatures[\rmfamily]{Ligatures=TeX,Scale=1}
\fi
% Use upquote if available, for straight quotes in verbatim environments
\IfFileExists{upquote.sty}{\usepackage{upquote}}{}
\IfFileExists{microtype.sty}{% use microtype if available
  \usepackage[]{microtype}
  \UseMicrotypeSet[protrusion]{basicmath} % disable protrusion for tt fonts
}{}
\makeatletter
\@ifundefined{KOMAClassName}{% if non-KOMA class
  \IfFileExists{parskip.sty}{%
    \usepackage{parskip}
  }{% else
    \setlength{\parindent}{0pt}
    \setlength{\parskip}{6pt plus 2pt minus 1pt}}
}{% if KOMA class
  \KOMAoptions{parskip=half}}
\makeatother
\usepackage{xcolor}
\usepackage[margin=1in]{geometry}
\usepackage{graphicx}
\makeatletter
\def\maxwidth{\ifdim\Gin@nat@width>\linewidth\linewidth\else\Gin@nat@width\fi}
\def\maxheight{\ifdim\Gin@nat@height>\textheight\textheight\else\Gin@nat@height\fi}
\makeatother
% Scale images if necessary, so that they will not overflow the page
% margins by default, and it is still possible to overwrite the defaults
% using explicit options in \includegraphics[width, height, ...]{}
\setkeys{Gin}{width=\maxwidth,height=\maxheight,keepaspectratio}
% Set default figure placement to htbp
\makeatletter
\def\fps@figure{htbp}
\makeatother
\setlength{\emergencystretch}{3em} % prevent overfull lines
\providecommand{\tightlist}{%
  \setlength{\itemsep}{0pt}\setlength{\parskip}{0pt}}
\setcounter{secnumdepth}{5}
\usepackage{titling}
\pretitle{\begin{center} \includegraphics[width=5in,height=13in]{front.jpg}\LARGE\\}
\posttitle{\end{center}}
\usepackage{fontawesome}
\usepackage[most]{tcolorbox}
\usepackage{xcolor}
\usepackage{sectsty}
\sectionfont{\color{olive}}
\usepackage{verbatim}
\ifLuaTeX
  \usepackage{selnolig}  % disable illegal ligatures
\fi
\IfFileExists{bookmark.sty}{\usepackage{bookmark}}{\usepackage{hyperref}}
\IfFileExists{xurl.sty}{\usepackage{xurl}}{} % add URL line breaks if available
\urlstyle{same} % disable monospaced font for URLs
\hypersetup{
  pdftitle={Reproducible and Collaborative Practices},
  pdfauthor={Patricia Menéndez},
  hidelinks,
  pdfcreator={LaTeX via pandoc}}

\title{Reproducible and Collaborative Practices}
\usepackage{etoolbox}
\makeatletter
\providecommand{\subtitle}[1]{% add subtitle to \maketitle
  \apptocmd{\@title}{\par {\large #1 \par}}{}{}
}
\makeatother
\subtitle{Tutorial 7}
\author{Patricia Menéndez}
\date{}

\begin{document}
\maketitle

{
\setcounter{tocdepth}{2}
\tableofcontents
}
\section*{Tutorial objectives:}
\begin{tcolorbox}
 \begin{itemize}
   \item Practice version control workflow.
   \item Ammend commits.
   \item Visit past commits and create branches from those.
   \item Merging branches and deal with conflicts.
   \item Practice pull requests.
   \item Use GitKraken as a tool to visualize trees.
   \item Use git reset and git revert.
 \end{itemize}
\end{tcolorbox}

\clearpage

\hypertarget{forks-pull-request-commits-and-inspecting-differences-between-commits-work-in-pairs}{%
\section{Forks, pull request, commits and inspecting differences between
commits (work in
pairs)}\label{forks-pull-request-commits-and-inspecting-differences-between-commits-work-in-pairs}}

During this exercise make sure that you use your terminal/cli and also
GitKraken to see the repo tree. Also keep an eye on your GitHub repo.

\begin{enumerate}
\def\labelenumi{\arabic{enumi}.}
\tightlist
\item
  Create a new \emph{public} GitHub repository called
  \emph{Tutorial7-XX} and replace \emph{XX} with your initials. Include
  a \emph{README.md} file.
\item
  Create a new branch called \emph{newbranchXX} and move the HEAD of
  your repo to the tip of \emph{newbranchXX}
\item
  In your new branch add a new file called \emph{exerciseXX.Rmd} and
  replace \emph{XX} with your initials.
\item
  Stage, commit and push the changes into the remote repo
\item
  Create a new folder in \emph{newbranchXX} called \emph{Images} and add
  \emph{Figure1.png} (you can find the figure in Moodle)
\item
  Stage, commit and push the changes into the remote repo.
\item
  Using \emph{git log} and \emph{git log --oneline} inspect the commits
  that you have made.
\item
  Exchange the details of your GitHub repo with your partner for the
  exercise.
\item
  Fork your partner's repository.
\item
  Clone locally your partner's repository \textbf{with all the
  branches}. \textbf{Hint:} See lecture slides page 36.
\item
  Continue working in your partner's branch and add the following to the
  \emph{exerciseXX.Rmd} YAML:
\end{enumerate}

\begin{tcolorbox}
\begin{verbatim}

title: "Reproducible and Collaborative Practices"
subtitle: "Tutorial 7"
author: "Your Name"
institute: "Department of Econometrics and Business Statistics"
output: 
 pdf_document:
   toc: true
   toc_depth: 2
   number_sections: true
   highlight: tango
header-includes: 
  - \usepackage{titling}
  - \pretitle{\begin{center}
    \includegraphics[width=5in,height=13in]{figs/front.jpg}
  - \posttitle{\end{center}}
  - \usepackage{fontawesome}
  - \usepackage[most]{tcolorbox}
  - \usepackage{xcolor}
  - \usepackage{sectsty}
  - \sectionfont{\color{olive}}
  - \usepackage{verbatim}
  
\end{verbatim}
\end{tcolorbox}

\begin{enumerate}
\def\labelenumi{\arabic{enumi}.}
\setcounter{enumi}{11}
\item
  Use \emph{git status} and \emph{git log --oneline} to inspect your
  repo.
\item
  Stage \emph{exerciseXX.Rmd}.
\item
  Unstage \emph{exerciseXX.Rmd}.
\item
  Stage, commit and push the changes into the remote repo.
\item
  Amend this last commit. \textbf{Hint:}

  \begin{itemize}
  \tightlist
  \item
    git commit --amend
  \item
    after that in your terminal you can use :q to get out of the text
    editor
  \item
    git push --force
  \item
    You can also right click on your last commit in the GitKraken tree
    and select \emph{edit commit message}.
  \end{itemize}
\item
  Add one new section into the \emph{exerciseXX.Rmd}.
\item
  Stage, commit and push the changes into the remote repo.
\item
  Inspect the differences between your last two commits. \textbf{Hint:}
  diff oldestcommit\_SHA .. HEAD --color -words
\item
  Go back to a previous commit of your choice and create a new branch
  from there. \textbf{Hint:} git checkout SHA/SHA1.
\item
  Checkout into the new branch and add a new section into
  \emph{exerciseXX.Rmd}. Then merge this new branch into your partner's
  branch.
\item
  Create a pull request to each other and accept the changes that your
  partner is suggesting for your repo dealing with any possible merging
  conflicts.
\item
  Once you have finished with this pull request merge the branch into
  master.
\end{enumerate}

\hypertarget{closer-look-to-commits-revert-to-previous-commits-and-reset-your-repository.}{%
\section{Closer look to commits, revert to previous commits and reset
your
repository.}\label{closer-look-to-commits-revert-to-previous-commits-and-reset-your-repository.}}

\begin{enumerate}
\def\labelenumi{\arabic{enumi}.}
\tightlist
\item
  Use GitKraken and inspect your previous exercise repo.
\item
  Inspect the tree and commits.
\item
  Create a new section in \emph{exerciseXX.Rmd}.
\item
  Stage, commit and push the changes.
\item
  Add a new \emph{latex list} inside the last section that you have
  created. \textbf{Hint:}

  \begin{verbatim} \begin{itemize} \item ... \end{itemize} \end{verbatim}
\item
  Stage, commit and push the changes.
\item
  Go back and find the SHA number of the second last commit and use
  \emph{git reset SHA}. Type \emph{git status} and observe what has
  changed. How many commits do you have now in your repo now?
\item
  In the terminal use \emph{git revert} to go back to one of your
  previous commits.
\end{enumerate}

\end{document}
